\documentclass[a4paper,twocolumn]{article}
\usepackage{template}
\usepackage{lipsum}  % For dummy text

\title{Academic Paper Title}
\author{Author Name\\
        Department of Computer Science\\
        University Name\\
        \texttt{email@university.edu}}
\date{}

\begin{document}

\maketitle

% Abstract and keywords span both columns
\twocolumn[
\begin{@twocolumnfalse}

\begin{abstract}
\noindent
\textbf{Abstract:} This paper presents a comprehensive study of game theory applications in modern computing systems. We analyze various strategic decision-making scenarios and propose novel algorithms for Nash equilibrium computation. Our experimental results demonstrate significant improvements over existing methods, with up to 40\% reduction in computational complexity. The findings have important implications for distributed systems, network protocols, and multi-agent coordination.
\end{abstract}

\keywords{game theory; Nash equilibrium; computational complexity; distributed systems; multi-agent systems; strategic decision-making}

\vspace{1em}
\end{@twocolumnfalse}
]

% Main content starts here in two-column format
\section{Introduction}

\lipsum[1-3]

\subsection{Motivation}

\lipsum[4]

\subsection{Contributions}

The main contributions of this paper are:
\begin{itemize}
    \item A novel algorithm for computing Nash equilibria in polynomial time
    \item Comprehensive experimental evaluation on real-world datasets
    \item Theoretical analysis of convergence properties
    \item Open-source implementation of the proposed methods
\end{itemize}

\section{Related Work}

\lipsum[5-7]

\subsection{Classical Game Theory}

\lipsum[8]

\subsection{Computational Methods}

\lipsum[9]

\section{Problem Formulation}

\lipsum[10-11]

\subsection{Formal Definition}

\lipsum[12]

\subsection{Complexity Analysis}

\lipsum[13]

\section{Proposed Method}

\lipsum[14-16]

\subsection{Algorithm Description}

\lipsum[17]

\subsection{Implementation Details}

\lipsum[18]

\section{Experimental Results}

\lipsum[19-21]

\subsection{Dataset Description}

\lipsum[22]

\subsection{Performance Metrics}

\lipsum[23]

\subsection{Comparison with Baselines}

\lipsum[24]

\section{Discussion}

\lipsum[25-26]

\section{Conclusion}

\lipsum[27-28]

\section*{Acknowledgments}

This work was supported by the National Science Foundation under Grant No. 123456. We thank the anonymous reviewers for their valuable feedback.

\begin{thebibliography}{99}

\bibitem{ref1}
Smith, J. and Johnson, A. (2023).
\textit{Game Theory Fundamentals}.
Journal of Theoretical Computer Science, 45(2), 123-145.

\bibitem{ref2}
Lee, K., Chen, M., and Wang, L. (2022).
\textit{Efficient Nash Equilibrium Computation}.
In Proceedings of the International Conference on Artificial Intelligence, pp. 567-578.

\bibitem{ref3}
Garcia, R. (2021).
\textit{Distributed Game-Theoretic Algorithms}.
ACM Computing Surveys, 53(4), Article 78.

\bibitem{ref4}
Patel, S. and Kumar, V. (2023).
\textit{Multi-Agent Systems and Strategic Interaction}.
Springer-Verlag, Berlin.

\bibitem{ref5}
Brown, T., Davis, E., and Wilson, M. (2022).
\textit{Complexity Analysis of Game-Theoretic Problems}.
Theoretical Computer Science, 789, 234-256.

\end{thebibliography}

\end{document}
