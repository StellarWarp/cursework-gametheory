\documentclass[a4paper,twocolumn]{article}
\usepackage{template}
\usepackage[UTF8]{ctex}  % Support for Chinese characters
\usepackage{lipsum}      % For dummy text

\title{示例论文标题\\Example Paper Title}
\author{作者姓名\\Author Name\\
        单位名称\\Institution Name}
\date{}

\begin{document}

% Title
\maketitle

% Abstract and keywords need to span both columns
% We use \twocolumn[...] to create full-width content at the top
\twocolumn[
\begin{@twocolumnfalse}

% Abstract
\begin{abstract}
\noindent
This is an example abstract that spans the full width of the page across both columns. The abstract provides a brief summary of the paper's content, methodology, and findings. It should be concise yet informative, typically between 150-250 words. The text uses a smaller font size than the main body text and has no first-line indentation as specified in the requirements.

这是一个跨越两栏全宽显示的摘要示例。摘要简要概述了论文的内容、方法和发现。它应该简洁而信息丰富,通常在150-250字之间。文本使用比正文稍小的字体,并且按照要求首行不缩进。
\end{abstract}

% Keywords
\keywords{game theory; Nash equilibrium; strategy; optimization; 博弈论; 纳什均衡}

\vspace{1em}
\end{@twocolumnfalse}
]

% Main content starts here in two-column format
\section{Introduction}
\lipsum[1-2]

\section{Background}
\lipsum[3-4]

\subsection{Related Work}
\lipsum[5]

\subsection{Theoretical Framework}
\lipsum[6]

\section{Methodology}
\lipsum[7-8]

\subsection{Experimental Setup}
\lipsum[9]

\subsection{Data Collection}
\lipsum[10]

\section{Results}
\lipsum[11-12]

\subsection{Analysis}
\lipsum[13]

\subsection{Discussion}
\lipsum[14]

\section{Conclusion}
\lipsum[15-16]

\section*{References}
\begin{thebibliography}{99}
\bibitem{ref1} Author A. (2023). \textit{Title of Paper}. Journal Name, 10(2), 123-145.
\bibitem{ref2} Author B. (2023). \textit{Another Paper Title}. Conference Proceedings, 456-478.
\end{thebibliography}

\end{document}
